\cleardoublepage
\phantomsection
\pdfbookmark{Sommario}{Sommario}
\begingroup
\let\clearpage\relax
\let\cleardoublepage\relax
\let\cleardoublepage\relax

\chapter*{Sommario}
Il presente elaborato documenta le attività svolte durante il periodo di stage (circa 300 ore) presso \textit{Halue S.r.l.} e descrive l'inserimento nel contesto di stage, la progettazione e la realizzazione di un prototipo di interfaccia conversazionale per un sito e-commerce nel settore skincare. Il lavoro comprende l'analisi dei requisiti, la progettazione architetturale di un sistema agentico integrato con modelli di linguaggio e retrieval-augmented generation (RAG), l'implementazione di connettori verso piattaforme (ad es. Shopify e Sanity), lo sviluppo del proof-of-concept e la verifica tramite test end-to-end.

\vspace{6pt}
\noindent\textbf{Struttura del documento.} Il testo è organizzato nei seguenti capitoli e sezioni principali:
\begin{enumerate}
  \item \textbf{Capitolo 1 — Azienda:} contesto aziendale, obiettivi, tecnologie e processi interni.
  \item \textbf{Capitolo 2 — Stage:} descrizione del progetto di stage, motivazioni, pianificazione, metodo di lavoro e strumenti adottati.
  \item \textbf{Capitolo 3 — Sviluppo:} requisiti funzionali e non funzionali, progettazione dell'architettura (storefront, integrazioni, database vettoriale), descrizione delle funzionalità implementate (chat, function calling, chaining, multi-agent, token streaming), test effettuati e problemi riscontrati.
  \item \textbf{Capitolo 4 — Conclusioni:} valutazione degli obiettivi raggiunti lato progetto e lato personale, retrospettiva e la valutazione dell'esperienza di stage.
  \item \textbf{Appendici e materiali complementari:} glossario, elenchi di figure e tabelle.
\end{enumerate}

\vspace{6pt}
\noindent\textbf{Convenzioni tipografiche.} Per garantire chiarezza e uniformità nella scrittura del documento sono state adottate le seguenti convenzioni tipografiche:
\begin{description}
  \item[Lingua:] italiano.
  \item[Carattere e corpo:] Times New Roman (o equivalente); corpo del testo 12\,pt; titoli dei capitoli 14\,pt (o come richiesto dal regolamento), interlinea 1.5.
  \item[Intestazioni:] numerazione gerarchica (es. 1, 1.1, 1.1.1), titoli dei paragrafi in grassetto.
  \item[Allineamento e margini:] testo giustificato; margini standard (ad es. 2.5\,cm su tutti i lati) salvo diversa indicazione.
  \item[Numerazione pagine:] numerazione romana per frontespizi e sommario (i, ii, ...); numerazione araba a partire dall'introduzione / Capitolo 1.
  \item[Figure e tabelle:] didascalie concise sotto la figura/tabella; numerazione progressiva (Figura 1.1, Tabella 2.1); riferimenti alle figure nel testo.
  \item[Codice e output:] font monospace (es. \texttt{Consolas} o \texttt{Courier New}), corpo 10\,pt, blocchi di codice delimitati e con caption descrittiva se presenti.
  \item[Abbreviazioni e termini tecnici:] alla prima occorrenza la forma estesa seguita dall'acronimo tra parentesi; uso coerente dell'acronimo in seguito.
  \item[Citazioni e bibliografia:] seguire lo stile indicato dal relatore (se non specificato, mantenere uno stile coerente come APA o IEEE in tutto il documento).
  \item[Note tipografiche:] termini in lingua straniera o parole chiave in corsivo; evitare uso eccessivo di maiuscole e colori non necessari.
\end{description}
\endgroup

\vfill
