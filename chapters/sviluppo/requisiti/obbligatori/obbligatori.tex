\subsection{Obbligatori}

\begin{table}[H]
    \centering
    \caption{Requisiti obbligatori}
    \label{tab:requisiti-obbligatori}
    \begin{tabularx}{\textwidth}{|p{0.15\textwidth}|>{\raggedright\arraybackslash}X|}
        \hline
        \textbf{ID} & \textbf{Descrizione} \\
        \hline
        RUO-1\label{ru:RUO-1} & L'utente deve poter interagire testualmente con un agente basato su intelligenza artificiale. \\
        \hline
        RUO-2\label{ru:RUO-2} & Il sistema deve fornire informazioni ausiliarie durante le ricerche effettuate dall'utente. \\
        \hline
        RUO-3\label{ru:RUO-3} & Il sistema deve permettere di aggiungere prodotti al carrello dell'utente. \\
        \hline
        RUO-4\label{ru:RUO-4} & Il sistema deve permettere di rimuovere prodotti dal carrello dell'utente. \\
        \hline
        RUO-5\label{ru:RUO-5} & Il sistema deve permettere di modificare le quantità o le varianti dei prodotti presenti nel carrello. \\
        \hline
        RUO-6\label{ru:RUO-6} & Il sistema deve predisporre il carrello per il processo di checkout. \\
        \hline
        RUO-7\label{ru:RUO-7} & Il sistema deve integrare un meccanismo di validazione che verifichi la coerenza delle richieste dell'utente con il contesto fornito. \\
        \hline
        RUO-8\label{ru:RUO-8} & L'utente deve poter navigare nel sistema senza difficoltà e senza necessità di conoscenze pregresse. \\
        \hline
    \end{tabularx}
\end{table}


%Sotto-sezione che riporterà la lista dei requisiti obbligatori discussi e studiati nel dettaglio.
