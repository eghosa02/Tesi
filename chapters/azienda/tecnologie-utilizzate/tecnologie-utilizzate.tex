\section{Tecnologie utilizzate}

Dal punto di vista delle piattaforme e degli strumenti operativi, le tecnologie principali rilevate sono le sguenti:
\begin{itemize}
%\item \emph{Shopify} impiegata soprattutto in progetti \emph{B2C} e per clienti di piccola/media dimensione che necessitano di tempi di rilascio rapidi e costi contenuti;
\item \emph{Salesforce Commerce Cloud} e relativi moduli \emph{Service Cloud} / \emph{Marketing Cloud} (suite enterprise per commercio, servizio clienti e automazione marketing): utilizzati per soluzioni enterprise (\emph{B2B}/\emph{B2C}) con esigenze avanzate di integrazione, automazione e gestione della \emph{customer experience};
%\item \emph{Sanity} e \emph{Storyblok} adottati per la gestione di contenuti dinamici in architetture \emph{decoupled}, con pubblicazione multicanale tramite \emph{API};
%\item \emph{Hydrogen} e \emph{Remix} usati quando l’esperienza utente richiede elevata customizzazione e performance;
\item \emph{Bloomreach} (piattaforma per ricerca avanzata e personalizzazione della customer journey): impiegato in progetti di scala;
\item Cloud e piattaforme di deployment: \emph{AWS}, \emph{Google Cloud} e \emph{Heroku} per hosting, provisioning di risorse scalabili e ambienti di \emph{staging} e \emph{production};
\item Strumenti di integrazione: approcci punto-a-punto e soluzioni \emph{iPaaS} (\emph{Integration Platform as a Service}: piattaforme per orchestrare integrazioni) per orchestrare flussi dati fra \emph{CRM}, \emph{ERP}, piattaforme e-commerce e \emph{CMS};
\item Strumenti di gestione progetto e comunicazione: \emph{Jira} per il tracciamento delle attività e \emph{Slack} per la comunicazione operativa quotidiana;
\item Controllo versione e pipeline: uso diffuso di \emph{Git}, pipeline di integrazione continua e \emph{test} automatici (unitari e d’integrazione) prima del rilascio.
\end{itemize}

A livello operativo queste tecnologie si traducono in pratiche concrete osservate però solo parzialmente durante lo stage: sviluppo su rami funzionali con \emph{code review}, 
creazione di ambienti di \emph{staging} e \emph{production} su cloud, esecuzione di \emph{test} automatici.

%Per la manutenzione e il supporto operativo si privilegia un approccio a livelli: monitoraggio dei servizi e dei log, 
%gestione dei \emph{ticket} tramite \emph{Jira} e interventi pianificati per aggiornamenti e patch. Nei progetti di maggiori 
%dimensioni si osserva una separazione tra team di sviluppo (feature delivery) e team di \emph{platform/operations} (monitoraggio, sicurezza, scaling), 
%mentre per clienti più piccoli le stesse persone spesso ricoprono più ruoli.

%Infine, la scelta delle tecnologie è stata chiaramente guidata dalla dimensione e dalla strategia del cliente: soluzioni \emph{Shopify} per rollout rapidi e basso \emph{TCO} 
%(\emph{Total Cost of Ownership}: costo totale di proprietà); architetture \emph{headless} e piattaforme \emph{Salesforce} per scenari enterprise e integrazioni complesse. 
%Ho inoltre riscontrato una forte propensione dell’azienda verso l’adozione di tecnologie emergenti legate all'\emph{AI} e alla personalizzazione, 
%inserite come componenti modulari (ad esempio agenti intelligenti collegati al \emph{CRM} o servizi di personalizzazione integrati con \emph{Bloomreach}). 
%In sintesi, l’ambiente tecnologico è eterogeneo e modulare, concepito per supportare sia progetti a rapido lancio sia soluzioni enterprise complesse, 
%con attenzione all’innovazione e alla scalabilità operativa.
%Sezione in cui verrà descritto il contesto produttivo in cui sono stato inserito, con particolare attenzione alle tecnologie operative viste adottare.
