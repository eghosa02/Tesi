\section{Processi interni}

Le informazioni qui riportate derivano da osservazione diretta del mio ambiente di lavoro e da brevi confronti con il tutor interno; 
durante i due mesi di stage ho svolto il progetto in larga parte in autonomia, perciò la mia percezione dei processi si basa per lo più da quanto accennato dal tutor aziendale
e da alcune conversazioni avuto con i membri del team.

Processo di sviluppo:
\begin{itemize}
\item assegnazione delle attività tramite \emph{Jira} (strumento di gestione dei progetti e del lavoro).
\item sviluppo locale e \emph{Versioning} con \emph{Git} (sistema di controllo versione); per i rilasci si utilizzano ambienti di prova e ambiente operativo, attivati tramite script forniti nel \emph{Repository} (archivio digitale che contiene codice, file e la cronologia delle modifiche, gestito da un sistema di versionamento);
\item comunicazione informale prevalente su \emph{Slack} e incontri quotidiani per allineamenti ; non ho partecipato agli \emph{stand-up meetings} data la natura autonoma del mio incarico;
\item gestione delle richieste post-rilascio tramite apertura di \emph{ticket} (voce nel sistema di tracciamento che rappresenta un'attività, un bug o una richiesta, con descrizione, stato, priorità e assegnatario) su \emph{Jira}.
\end{itemize}

processo di organizzazione:
\begin{itemize}
    \item le discussioni relative alle task (dubbi, proposte) avviene nellle sezioni relative su \emph{Jira}.
    \item le discussioni di stampo più casuali (proposte di meeting, rchieste di conronto) avvengono tramite \emph{Slack}.
    \item ogni comunicazione di stampo più formale non orientata ad una task nello specifico (contesto burroscratico) avviene tramite email.
\end{itemize}

Non sono stato coinvolto in attività legate al loro processo di manutenzione, né posso dire di averne discusso con il tutor aziendale o con i membri del team.