\section{Progetto di stage}

%descrizione stage
Il progetto di \emph{stage} ha come scopo la realizzazione di un prototipo di interfaccia conversazionale
avanzata per il sito eCommerce di un brand di skincare, basata su architettura agentica.
Lo \emph{stage} ha avuto come compito, l'intera progettazione, sviluppo e test di un sistema conversazionale che, tramite l'utilizzo
di \emph{Large Language Models} (\emph{LLM}), \emph{Retrieval-Augmented Generation} ((\emph{RAG})), agenti \emph{software} e \emph{server} \emph{MCP}(), fosse in grado di:
Dialogare con il cliente finale per rispondere a domande su brand, prodotti, routine skincare e quant' altro circa il mondo della \emph{skin-care}.
Interagire con il backend Shopify per reperire dati o eseguire azioni.
Sfruttare architetture modulari a microservizi per la scalabilità del sistema.

%problematiche applicative da voler affrontare con il tirocinio
Tale progetto ha come obbiettivo, riuscire a ridurre notevolmente il carico informativo richiesto all'utente per poter navigare sul sito, per poter consultare e decidere sui prodotti esposti.
Servizzi come nel caso specifico, quello della \emph{skin-care} peccano di un effetto \emph{overwhelming} indotto all'utente dovuta alla relativa grande quantità d'informazioni da dover assorbire
anche al solo fine di poter consultare i prodotti presenti nell \emph{e-commerce}.
Il prgetto dunque mira ha abbattere tale barriera, integrando un sistema intelligente capace di guidare l'utente attraverso la descrizione dei vari prodotti, integrazioni con possibili routine, 
compatibilità tra prodotti, prodotti mirati per le problematiche riportate testualmente dall'utente, ecc... megliorando di conseguenza anche l' \emph{User Experience} (\emph{UX}).


%obiettivi specifici prefissati e i vincoli operativi e temporali associati
Lo \emph{stage} per sua natura non è stato rigido sui vincoli tecnologici in quanto campo relativamente nuovo su cui sperimentare; sono stati invece posti obiettivi circa la produzione di 
documenti dediti a manifestare uno studio per poi discutere circa le tecnologie da approfondire, la progettazione architetturale e del \emph{PoC} in se e per se.
L'unico vincolo definibile come tale è stato la scelta di tecnologie tali da essere a favore di una facile distribuzione e dunque uscita nel mercato al fine di costringermi a lavorare in un 
ottica più prontata al lato dell'utente e economica più che ad una tendenza innata a reinventare la ruota tipica dei neo-laureati.
Riguardo gli obiettivi, per lo \emph{stage} sono stati posti con un ottica strettamente pessimistica consci del fatto della non chiara deifficolta di essi, in fatti in fase di \emph{stage},
sono stati inseriti più obiettivi da soddisfare man mano che ultimavo quelli precedentemente dati.

% rapporto tra la proposta di stage e la strategia più ampia dell'ente ospitante in materia di innovazione ()
Tale approccio 'morbido' sui vincoli e obiettivi non ha reso il progetto di per sè semplice, invece a fatto emergere in chiaro quele fosse l'ottica da avere sul progetto
e come approcciarlo.
Il seguente progetto ha richiesto necessariamente una fase di 'gioco' su un (\emph{playground}) al fine di comprendere meglio sul lato pratico le tecnologie più famose circa il contesto
sui cui ha mirato il progetto e poterle mettere a confronto stile dei pro e contro utili per discriminare alcune tecnologie a favore di altre.
Tale approccio sperimentale è stato per me una chiara evidenza di un approccio all'innovazione da parte del tutor aziendale anche per il suo approccio nel verificare parallelamente tecnologie
su carta simil potenti a quelle che sceglievo io al fine di verificare concretamente possibili \emph{trade-off} significativi che senza sperimentazione non sarebbero potuti emergere,
ottimizzando così il tempo dedicato alla sperimentazione e di conseguenza alla progettazione dell' architettura finale del \emph{PoC}

% attività di supporto previste prima, durante e dopo il periodo di tirocinio




%Sezione in cui verrà illustrato il progetto di stage ricevuto, esplicitando le problematiche applicative che l'organizzazione intende affrontare con il tirocinio, 
%gli obiettivi specifici prefissati e i vincoli operativi e temporali associati. Verrà inoltre evidenziato il rapporto tra la proposta di stage e la strategia più ampia dell'ente ospitante 
%in materia di innovazione (con riferimento al ruolo e alla posizione assunta dal tutor aziendale emerse nel primo incontro) 
%nonché le attività di supporto previste prima, durante e dopo il periodo di tirocinio.
%Qui concluderò la trattazione del punto 2 (perché) riportato nel file Struttura relazione finale.pdf.