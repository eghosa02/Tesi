\section{Descrizione generale}

Indico come ho raccolto le informazioni e quanto le ritengo attendibili, prima di entrare nel dettaglio. Le osservazioni che seguono si basano principalmente sulla partecipazione 
a \emph{meeting} (incontri di lavoro) con il tutor e su osservazioni effettuate sul posto, osservando i membri del team durante il lavoro. 
Per quanto riguarda le tecnologie e i processi quotidiani che ho visto direttamente, considero le informazioni affidabili; per decisioni strategiche e dati aziendali che mi sono 
stati riferiti in parte oralmente e in parte raccolti indirettamente, l'accuratezza è medio-bassa; per elementi finanziari o metriche non accessibili rimane bassa.

Halue S.r.l. è una società benefit di consulenza tecnica orientata all'adozione di soluzioni basate sull'intelligenza artificiale e alla fornitura di servizi a clienti sia 
\emph{B2C} (business-to-consumer: azienda verso consumatore) sia \emph{B2B} (business-to-business: azienda verso azienda). 
L'attività dell'azienda si diramano in vari servizzi tra i quali: la progettazione e gestione di soluzioni \emph{e-commerce} (commercio elettronico), l' implementazione di sistemi di 
gestione delle relazioni con i clienti (\emph{CRM}, Customer Relationship Management), servizi legati all'IA (es. sviluppo di agenti intelligenti) 
e percorsi di formazione rivolti ai clienti.
%contesto organizzativo e produttivo

\medskip
\noindent\textbf{Contesto organizzativo}
L'organizzazione aziendale riscontrata è tipica di una piccola realtà di consulenza dove in un unico team i propri membri alternano i ruoli predefiniti di verificatore e sviluppatore.
Viene gestita un organizzazione che concilia attività sia di stampo ibrido che totalmente da remoto tramite degli \emph{stand-up meetings} (incontro con cadenza periodica per allineare il team) quotidiani in cui ogni membro del team riporta il proprio avanzamento.
La comunicazione tra i membri del team si sviluppa su diversi layer in parte preferenziale ma per lo più al fine di organizzare le conversazioni per rendere ordinato il processo di richiesta, ricerca e inoltro delle informazioni:

\begin{itemize}
\item richieste di aiuto minori possono essere mosse verbalmente, mentre richieste più mirate e specifiche passano presso uno strumento di gestione dei progetti in cui ogni membro ha il suo ramo lavorativo di riferimento in cui \emph{postare}
i propri dubbi o proposte.
\item richieste di \emph{meeting} (riunione tra persone per discutere, coordinare o decidere su attività, problemi o informazioni) o chiarimento con necessità di porre un accordo temporale, vengono mosse tramite il rispettivo social di riferimento usato unanimemente da tutto il team.
\item in fine le richieste di stampo più burrocratico e/o legislativo vengono fatte tramite gmail.
\end{itemize}

\medskip
\noindent\textbf{Contesto produttivo}
Lato produzione, viene favorito la rapida inmissione nel mercato, con architetture che favoriscano dunque in basso ostacolo sul fronte del \emph{deployment} (messa in distribuzione del software in un ambiente operativo).
Ho potuto ricevere riscontro sull'uso dell' approccio \emph{Agile} (metodologia di lavoro iterativa e incrementale con rapido adattamento ai cambiamenti dei requisiti), con gestione tramite \emph{backlog} (elenco prioritizzato di attività) e strumenti di tracciamento; 
anche la comunicazione interna è strutturata a seconda delle necessità. Ho osservato inoltre la presenza di 
ambienti separati per il \emph{testing} (collaudo) e  per la produzione, e sono venuto a conoscenza dell'esistenza di procedure di rilascio automatizzate adattate ai singoli progetti.
Si alternano inoltre attività di \emph{delivery} (consegna/erogazione del servizio) su progetti su misura e attività di supporto e manutenzione post-consegna. 

\medskip
\noindent\textbf{Clientela}
%clientela
La clientela a cui la società si rivolge, sulla base dei progetti osservati e in linea con il mio progetto di stage, va dalle piccole e medie imprese con esigenze di commercio 
elettronico fino a grandi committenti che richiedono integrazioni complesse e soluzioni \emph{CRM} articolate; sono presenti commesse in ambito privato (\emph{retail}, 
distributori, operatori \emph{B2B}) e interventi rivolti a processi interni di organizzazioni di maggiori dimensioni.
%propensione all'innovazione

\medskip
\noindent\textbf{Propensione all'innovazione}
La propensione all'innovazione da parte, dell'azienda è percepibile dai seguenti fattori: attenzione alla formazione interna, approcci architetturali moderni e sperimentazione su progetti legati all'IA.
Tale propensione è comunque bilanciata da un approccio pragmatico al fine di privilegiare la stabilità quando richiesto da vincoli operativi o di tempo. 

% Sezione che riporterà la descrizione generale dell'azienda.
% Qui descriverò parzialmente il punto 1 (dove) riportato nel file Struttura relazione finale.pdf che terminerà nella sezione successiva.
