\subsection{Comunicazione con shopify}
% descrizione delle funzione per cui la comunicazione con shopify serva (ritornare il carrello, operazioni di rimozione, aggiunta e update al carrello, link al checkout, ecc...)
La comunicazione con Shopify espone operazioni e-commerce:
interrogare il carrello (prodotti, quantità, sottototale) via Storefront API
aggiungere prodotti al carrello (slug o variant ID)
aggiornare le quantità delle righe
rimuovere righe per completare il carrello finale
ottenere il link di checkout
Le risposte includono lo stato aggiornato del carrello per mostrarlo all’utente.
% descrivere come tali richieste vengano soddisfatte tramite le function-calling da parte dell'agente
Le richieste sono soddisfatte tramite \emph{function-calling}. I tool Python (mutation/query) sono registrati nell’agente. 
L’LLM seleziona il tool in base all’intento (es. “aggiungi prodotto Y”, “mostra carrello”, “link checkout”). 
Le mutazioni usano \emph{GraphQL} invocate via \texttt{requests.post} con autenticazione via \texttt{X-Shopify-Storefront-Access-Token}. 
I risultati JSON sono processati e passati all’agente per la risposta in linguaggio naturale.
% elenca i tool specifici e cosa fanno
Tool esposti:
\verb|get_cart_shopify_agent(cart_id: str)|: \emph{query GraphQL} per id, titoli, quantità; ritorna sottototale.
\verb|add_cart_shopify_agent_by_slug(cart_id: str, items: list)|: trasforma slug in variant ID (endpoint Shopify JSON); aggiunge con \texttt{cartLinesAdd} e ritorna carrello aggiornato.
\verb|add_cart_shopify_agent_by_variantIds(cart_id: str, items: list)|: aggiunge direttamente da variant ID; ritorna carrello aggiornato.
\verb|update_cart_lines_shopify_agent(cart_id: str, items: list)|: mutazione \texttt{cartLinesUpdate} per righe esistenti; ritorna carrello aggiornato.
\verb|remove_cart_lines_shopify_agent(cart_id: str, line_ids: list)|: mutazione \texttt{cartLinesRemove}; ritorna carrello aggiornato.
\verb|get_cart_line_ids_shopify_agent(cart_id: str)|: query per gli ID riga del carrello.
\verb|get_checkout_link(cart_id: str)|: query per l’URL di checkout.
\verb|get_variant_ids_by_product_id(product_ids: list[int])|: conversione product ID → variant ID (fino a 100 varianti); supporto per prodotti multi-variante.
Funzioni interne:
\verb|get_variantId(product_names: list[str])|: recupero slug via JSON API Shopify; mapping a variant ID.
% Sotto-sezione che riporterà la descrizione delle comunicazione con le API di Shopify da parte del sistema.
% Riepilogando: le API (Storefront GraphQL e JSON) sono incapsulate nei tool; l’agente sceglie il tool corretto, le invocazioni usano GraphQL con POST e Bearer; le risposte JSON espongono lo stato aggiornato del carrello; il flusso garantisce coerenza tra dati Sanity, Shopify e UI Hydrogen.
