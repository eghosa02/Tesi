\section{Progetto di stage}

Il progetto di \emph{stage} ha avuto come obiettivo principale la realizzazione di un prototipo di interfaccia conversazionale avanzata per il sito \emph{e-commerce} di un brand di \emph{skin-care},
fondata su un’architettura agentica e orientata all’integrazione con i sistemi già in uso presso l’azienda ospitante. 
L’iniziativa si inserisce all’interno della visione strategica di innovazione dell’organizzazione, che mira a potenziare l’esperienza digitale dei propri clienti e a 
esplorare le potenzialità offerte dalle più recenti tecnologie basate su intelligenza artificiale generativa.

Lo \emph{stage} ha previsto un percorso completo che includeva analisi preliminare, progettazione, sviluppo e test di un sistema conversazionale in grado di 
interagire in linguaggio naturale con gli utenti del sito, fornendo risposte pertinenti, contestuali e coerenti con i contenuti aziendali. Il prototipo è stato progettato 
per utilizzare un insieme integrato di tecnologie, tra cui \emph{Large Language Models} (\emph{LLM}), \emph{Retrieval-Augmented Generation} (\emph{RAG}), agenti \emph{software}, 
\emph{server} \emph{MCP} e le \emph{API} di \emph{Shopify}. Tali strumenti consentono all’agente di dialogare con il cliente finale per rispondere a domande su brand, prodotti, 
routine di \emph{skin-care} e altri argomenti affini, interagendo con il backend per reperire informazioni o eseguire azioni, e sfruttando architetture modulari a 
microservizi per garantire scalabilità e manutenibilità del sistema.

Dal punto di vista applicativo, il progetto è stato concepito per affrontare un problema concreto del dominio della \emph{skin-care}: la complessità informativa 
che spesso ostacola l’utente nella navigazione di un catalogo ampio e specializzato. Il fenomeno di \emph{overwhelming}, ossia il sovraccarico cognitivo derivante 
dall’eccesso di informazioni, rappresenta una delle principali sfide per gli \emph{e-commerce} del settore. Il sistema proposto mira a ridurre tale complessità, 
fungendo da mediatore intelligente capace di comprendere le richieste dell’utente, fornire spiegazioni, suggerire routine personalizzate, verificare la compatibilità 
tra prodotti e individuare soluzioni mirate alle esigenze espresse. In questo modo, il progetto contribuisce al miglioramento dell’esperienza utente (\emph{User Experience}, \emph{UX}) 
e alla valorizzazione del rapporto di fiducia tra cliente e brand.

Lo \emph{stage}, per sua natura esplorativa, non è stato rigidamente vincolato a un insieme fisso di tecnologie o metodologie, ma ha lasciato spazio alla 
sperimentazione e all’adattamento progressivo. I vincoli principali hanno riguardato l’utilizzo di \emph{Shopify} per la gestione del canale \emph{e-commerce} e di \emph{Sanity} 
come \emph{CMS}, in quanto piattaforme già adottate dall’azienda ospitante e quindi imprescindibili per garantire compatibilità e continuità con l’infrastruttura esistente. 
L’obiettivo era infatti quello di sviluppare un prototipo realistico e facilmente integrabile, capace di dimostrare la fattibilità tecnica e il potenziale valore strategico 
di un sistema conversazionale intelligente.

Dal punto di vista operativo e temporale, il progetto è stato suddiviso in più fasi: una prima fase di studio delle tecnologie e di definizione dei requisiti, 
una seconda fase di progettazione architetturale e sperimentazione su un \emph{playground}, e infine una fase di sviluppo del \emph{PoC} vero e proprio, con attività di test e validazione. 
Il lavoro si è svolto in stretto coordinamento con il tutor aziendale, che ha svolto un ruolo chiave di guida e confronto tecnico-strategico, garantendo la coerenza tra 
le scelte progettuali e la visione di innovazione perseguita dall’azienda. In particolare, il tutor ha contribuito alla definizione del perimetro di sperimentazione e 
ha condotto in parallelo analisi comparative di tecnologie affini a quelle testate, al fine di valutare concretamente i possibili \emph{trade-off} e ottimizzare i tempi di sviluppo.

L’approccio adottato è stato volutamente “morbido” riguardo agli obiettivi, permettendo di ampliare progressivamente il perimetro del progetto in base ai risultati ottenuti. 
Tale flessibilità ha favorito una gestione dinamica dello \emph{stage}, nella quale obiettivi iniziali più cauti sono stati progressivamente integrati con traguardi più ambiziosi, 
mantenendo però un’attenzione costante alla fattibilità e all’efficienza economica. Il vincolo principale, oltre alla compatibilità tecnologica, 
è stato quello di privilegiare soluzioni semplici, modulabili e pronte alla distribuzione, evitando la tendenza a “reinventare la ruota” e promuovendo invece un approccio pragmatico, 
centrato sull’utente e orientato al valore concreto per l’azienda.

Questo progetto di \emph{stage} riflette pienamente la strategia di innovazione dell’organizzazione ospitante, che vede negli \emph{stage} 
e nei progetti sperimentali uno strumento essenziale per esplorare nuove tecnologie, testarne le applicazioni in contesti reali e formare competenze interne. 
L’azienda, infatti, adotta una visione di innovazione incrementale, basata su iterazioni successive di apprendimento e validazione, piuttosto che su approcci 
radicali e ad alto rischio. In questa prospettiva, lo \emph{stage} non rappresenta un episodio isolato, ma una parte di un processo più ampio di evoluzione tecnologica, 
volto a migliorare la competitività del brand attraverso soluzioni digitali sempre più personalizzate e intelligenti.

Le attività di supporto sono state organizzate in modo da garantire continuità e collaborazione costante: briefing iniziali per l’allineamento sugli obiettivi, 
incontri periodici di revisione, sessioni di confronto tecnico e momenti di verifica intermedia dello stato di avanzamento. Al termine dello \emph{stage}, 
è stata prevista la consegna di tutta la documentazione tecnica e dei risultati del \emph{PoC}, in modo da permettere all’azienda di proseguire in autonomia lo sviluppo e 
valutare la possibilità di una futura industrializzazione del prototipo.

In conclusione, il progetto di \emph{stage} ha rappresentato un’occasione concreta per esplorare l’applicazione dell’intelligenza artificiale conversazionale 
in un contesto reale di \emph{e-commerce}, contribuendo non solo al miglioramento dell’esperienza utente ma anche alla definizione di una visione strategica 
più ampia per l’azienda ospitante, in linea con la sua continua ricerca di innovazione tecnologica e sostenibilità digitale.








% attività di supporto previste prima, durante e dopo il periodo di tirocinio




%Sezione in cui verrà illustrato il progetto di stage ricevuto, esplicitando le problematiche applicative che l'organizzazione intende affrontare con il tirocinio, 
%gli obiettivi specifici prefissati e i vincoli operativi e temporali associati. Verrà inoltre evidenziato il rapporto tra la proposta di stage e la strategia più ampia dell'ente ospitante 
%in materia di innovazione (con riferimento al ruolo e alla posizione assunta dal tutor aziendale emerse nel primo incontro) 
%nonché le attività di supporto previste prima, durante e dopo il periodo di tirocinio.
%Qui concluderò la trattazione del punto 2 (perché) riportato nel file Struttura relazione finale.pdf.