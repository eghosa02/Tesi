\section{Metodo di lavoro}

Per lo sviluppo del progetto, in accordo con il tutor aziendale è stata scelta una modalità ibrida di lavoro con  1-2 giorni alla settimana in presenza, il resto da remoto,
ciò per farmi entrare nello stesso ambiente lavorativo degli altri membri del team che adottano la stessa modalità.
Ai fini comunicativi compatibili con la modalità, ho utilizzato \emph{salck} per le comunicazioni con il tutor aziendale riguardanti richieste di consigli, comunicare entro fine settimana i giorni di presenza per quella successiva
e invio di link e documentazione da ambo le parti.
Per il lavoro in presenza ho usufruito di un mio \emph{pc} portatile.

Il flusso di lavoro da adoperare mi è stato in parte esplicato e in parte discusso e deciso richiedendo la mia opinione.
Per ogni fase definita a priori nel piano di lavoro, avrei dovuto fornire un documento riportante gli sviluppi fatti durante quella fase, 
in particolare:
Un documento per l'analsi del dominio, che per fini pratici si è tramutato in una mostra di un mio playground per dimostrare uno studio ccirca le tecnologie
principali che sarebbero poi state approfondite anche in fase di sviluppo tra cui l' uso di librerie per i \emph{Large Language Models} (\emph{LLM}), 
\emph{RAG}, agenti \emph{software}, \emph{MCP} e le \emph{API} di \emph{Shopify}.

Un documento nel quale avrei discusso la scelta di particolari soluzioni tecnologie per le relative tecnologie rispetto ad altre, motivandole con il fine di convincere
il tutor aziendale della convenienza temporale e fattibilità delle soluzioni da me proposte.

Un documento di specifica tecnica per documentare il funzionamento del \emph{PoC} in fase conclusiva del progetto

La \emph{repository} nella quale avrei ordinatamente caricato il progetto seguendo un' alberatura senzata e ragionata con il relativo \emph{file} \emph{Readme.md}. 

%interazioni con il tutor aziendale



%Sezione che riporterà il flusso di lavoro uilizzato per lo sviluppo del progetto in accordo con il tutor aziendale.
%Verranno riportati pianificazione, interazioni con il tutor aziendale, revisioni di progresso, uso di diagrammi,
%tecniche di analisi e tracciamento dei requisiti, strumenti di verifica, ecc.
%Qui descriverò il punto 3.a (cosa e come) riportato nel file Struttura relazione finale.pdf.
