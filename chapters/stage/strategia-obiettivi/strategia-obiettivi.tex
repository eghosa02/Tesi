\section{Strategia}% e obiettivi di stage

Sono entrato nel periodo di stage durante una fase di sviluppo da parte del \emph{team}, riguardo un \emph{re-breanding} del sito \emph{e-commerce} appartenente ad 
un' azienda chiamata \emph{comfortzone}.
essendo l' azienda ospitante caratterizzata anche da fasi di ricerca e sviluppo, l' idea di sviluppo è stata riguardante una possibile integrazione di un sistema intelligente
all'interno della loro infrastruttura composta da \emph{e-commerce} e \emph{blog}.
L' idea è stata quella di:
Studiare le principale tecnologie riguardanti il mondo degli agenti IA, relativi \emph{framework}, \emph{design pattern} agentici.
Come poterli integrare in un' architettura originale caratterizzata da una facile e rapida uscita a mercato (fase di \emph{deploy}).
Produrre un \emph{PoC} da proporre all' azienda di riferimento (\emph{comfortzone}).

La strategia di stage è orientata a risolvere l'effetto \emph{overwelming} subito da parte degli utenti che s'interfacciano su nuovi mondi (come la \emph{skin-care} nel seguente contesto), tanto più invece a portare qualcosa di nuovo
in line con la propensione all' inovazione da parte dell' azienda ospitante.

Da quanto mi è stato riportato dal tutor aziendale, non è stato definito a priori una particolare \emph{road map} riguardante la seguente strategia circa possibili sviluppi
tramite ulteriori stage, quanto più un' approccio sperimentale in un campo su cui non c'è molto su cui basarsi per avere un riscontro sulle \emph{best practices}.

%farlo correggere da chat gpt mantenendo gli inglesismi
%farlo espandere da chatgpt mantenendo gli inglesismi
%creare prompt per immagine illustrative
%togliere sfondo all'immagine
%inserire l'immagine



%Sezione che riporterà come lo stage si inserisce nella visione strategica da parte dell'aziendale (e dunque la  propensione dell’azienda per l’innovazione).
%Qui descriverò parzialmente il punto 2 (perché) riportato nel file Struttura relazione finale.pdf. e lo concluderò nella sezione successiva.
