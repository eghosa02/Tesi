\subsection{Requisiti}

\begin{table}[H]
    \centering
    \caption{Obiettivi obbligatori}
    \begin{tabularx}{\textwidth}{@{}l X@{}}
      \toprule
      \textbf{Codice} & \textbf{Descrizione} \\
      \midrule
      O01 & Realizzare un'interfaccia conversazionale funzionante per eCommerce \textit{skincare}. \\
      O02 & Integrare la conversazione con sorgenti informative tramite \textbf{RAG} (Retrieval-Augmented Generation). \\
      O03 & Collegare la conversazione a task automatizzati su Shopify tramite agent \textbf{MCP}. \\
      \bottomrule
    \end{tabularx}
  \end{table}
  
  \vspace{8pt}
  
  \begin{table}[H]
    \centering
    \caption{Obiettivi desiderabili}
    \begin{tabularx}{\textwidth}{@{}l X@{}}
      \toprule
      \textbf{Codice} & \textbf{Descrizione} \\
      \midrule
      D01 & Abilitare una gestione dinamica dei contenuti (es. aggiornamenti promozionali, modifiche alle routine skincare). \\
      D02 & Adottare meccanismi di fallback per gestire l'incertezza del modello (es. risposte alternative, escalation a operatore umano). \\
      \bottomrule
    \end{tabularx}
  \end{table}
  
  \vspace{8pt}
  
  \begin{table}[H]
    \centering
    \caption{Obiettivi facoltativi}
    \begin{tabularx}{\textwidth}{@{}l X@{}}
      \toprule
      \textbf{Codice} & \textbf{Descrizione} \\
      \midrule
      F01 & Integrazione con strumenti di analytics per tracciare le conversazioni e misurare metriche di performance. \\
      F02 & Prototipo multilingua (EN / IT) per supportare utenti in inglese e italiano. \\
      F03 & Sperimentazione su front-end mobile/web per valutare usabilità e comportamento cross-device. \\
      \bottomrule
    \end{tabularx}
  \end{table}

%Sotto-sezione che riporterà la lista dei requisiti per il progetto presenti nel piano di lavoro.
