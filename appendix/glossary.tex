\cleardoublepage
\chapter*{Glossario}
\addcontentsline{toc}{chapter}{Glossario}
\markboth{Glossario}{Glossario}


\section*{A}
\begin{description}
    \item[API] \textbf{Application Programming Interface} In informatica il termine indica ogni insieme di procedure disponibili al programmatore, di solito raggruppate a formare un set di strumenti specifici per l'espletamento di un determinato compito all'interno di un certo programma. La finalità è ottenere un'astrazione tra l'hardware e il programmatore o tra software a basso e quello ad alto livello, semplificando così il lavoro di programmazione.
    \item[Agile] Approccio allo sviluppo del software e alla gestione progetti basato su iterazioni brevi, feedback continui e adattamento ai cambiamenti. Include metodologie come Scrum e Kanban.
    \item[AI] \textbf{Artificial Intelligence (Intelligenza Artificiale)} Insieme di tecniche e algoritmi che consentono a un sistema di simulare capacità cognitive tipiche dell'essere umano: apprendimento, ragionamento, percezione e interazione.
    %\item[Alerting] \textbf{Alerting} Sistema di notifiche automatiche che avvisa quando si verificano anomalie, errori o condizioni critiche in applicazioni e infrastrutture.
\end{description}

\section*{B}
\begin{description}
    \item[B2C] \textbf{Business-to-Consumer} Modello di business in cui un'azienda vende prodotti o servizi direttamente al consumatore finale. Esempi tipici: e-commerce al dettaglio, servizi in abbonamento per privati.
    \item[B2B] \textbf{Business-to-Business} Modello di business in cui le transazioni avvengono tra imprese; prodotti o servizi rivolti ad altre aziende (software aziendale, forniture industriali, consulenza).
    \item[Backlog] Elenco prioritizzato di attività, requisiti o user story che rappresentano il lavoro da svolgere su un prodotto o progetto. Nel contesto Agile esistono tipicamente un product backlog (tutte le feature/attività previste) e uno sprint backlog (selezione per uno sprint).
    \item[Branch] Linea di sviluppo separata in un sistema di controllo versione (es. Git). I branch permettono di lavorare contemporaneamente su feature, fix o release senza intaccare il ramo principale; tipiche operazioni correlate: commit, merge, rebase e pull request.
    %\item[Build] \textbf{Build} Processo di compilazione e packaging del codice sorgente in un artefatto eseguibile o distribuibile (ad esempio un file binario, una libreria, un container).
\end{description}

\section*{C}
\begin{description}
    \item[CI/CD] \textbf{Continuous Integration / Continuous Delivery} Insieme di pratiche che prevedono integrazione continua del codice tramite test automatici (CI) e rilascio frequente e affidabile del software (CD).
    \item[CMS] \textbf{Content Management System} Piattaforma software che permette di creare, modificare e pubblicare contenuti digitali (pagine web, articoli, media) con minime conoscenze tecniche; esempi: WordPress, Drupal.
    %\item[CRM] \textbf{Customer Relationship Management} Sistema e strategia per gestire le relazioni con i clienti: archiviazione anagrafiche, storico interazioni, gestione vendite e opportunità, automazione marketing e supporto post-vendita. Esempi: Salesforce, HubSpot.
    \item[Cloud-native] Architettura e modalità di sviluppo che sfrutta appieno le caratteristiche del cloud (scalabilità, resilienza, containerizzazione, microservizi).
    %\item[Compliance] \textbf{Compliance} Conformità a normative, regolamenti o standard di settore (es. GDPR, ISO 27001).
    \item[Container] Tecnologia che consente di pacchettizzare applicazioni e dipendenze in un'unità isolata e portabile, eseguibile in diversi ambienti.
    \item[Code review] Revisione del codice sorgente da parte di uno o più membri del team, con l’obiettivo di migliorare qualità, leggibilità e sicurezza.
    \item[Client-side] Termine che indica le operazioni o i processi eseguiti sul dispositivo dell’utente finale (browser, smartphone, computer), piuttosto che sul server remoto.  
    Nel contesto delle applicazioni web, il codice *client-side* — tipicamente scritto in linguaggi come JavaScript, HTML e CSS — gestisce l’interfaccia utente, le interazioni e parte della logica applicativa.  
    \item[Chunking] Processo di suddivisione di un testo, documento o insieme di dati in porzioni più piccole e gestibili, dette *chunk*.  
    Nel contesto dell’intelligenza artificiale e del *Retrieval-Augmented Generation (RAG)*, il chunking serve a migliorare la precisione nel recupero delle informazioni: i documenti vengono divisi in segmenti con senso compiuto, che possono poi essere indicizzati e richiamati quando il modello genera una risposta.  
\end{description}

\section*{D}
\begin{description}
    \item[Delivery] Consegna di un prodotto, servizio o funzionalità al cliente o all'ambiente di produzione; include fasi di build, test, integrazione e deployment. In ambito Agile si parla spesso di "continuous delivery" per indicare rilascio frequente e affidabile.
    \item[Deploy] \textbf{Deployment} Processo che porta il codice da un ambiente di sviluppo/testing all'ambiente di produzione (o ad un ambiente target). Comprende build, configurazione, esecuzione di script di migrazione, verifica post-deploy e possibilità di rollback. Spesso automatizzato tramite pipeline CI/CD.
    \item[Deliverables] Risultati tangibili, documenti o prodotti che devono essere consegnati al termine di un’attività, fase o progetto. Possono includere report, prototipi, software, presentazioni o altri output concordati, e rappresentano la misura concreta del progresso e del completamento del lavoro pianificato.
\end{description}

\section*{E}
\begin{description}
    \item[E-commerce] Abbreviazione di *electronic commerce*, indica l’acquisto e la vendita di beni o servizi attraverso piattaforme digitali, come siti web o app. 
    \item[Embedding] Rappresentazione di dati (tipicamente testo, ma anche immagini o suoni) sotto forma di vettori numerici in uno spazio multidimensionale.  
    Nel contesto del linguaggio naturale, gli *embeddings* permettono di catturare relazioni semantiche tra parole o frasi: termini con significati simili sono rappresentati da vettori vicini nello spazio.  
    
\end{description}

\section*{F}
\begin{description}
    \item[Feature-branch] Ramo di un repository di controllo versione creato per sviluppare una specifica funzionalità in modo isolato dal ramo principale (main/master). Permette al team di lavorare in parallelo senza interferire con il codice stabile, facilitando testing, revisione del codice (code review) e integrazione controllata tramite merge o pull request.
    \item[Feedback] Informazione restituita a un individuo o a un team sul lavoro svolto, utile per migliorare prodotti, processi e comportamenti.
    \item[Frontend] Parte visibile di un’applicazione o sito web con cui interagisce direttamente l’utente (UI/UX).
\end{description}

\section*{G}
\begin{description}
    \item[Git] Sistema distribuito di controllo versione, utilizzato per tracciare modifiche al codice sorgente e coordinare il lavoro tra più sviluppatori.
    \item[GROQ] Acronimo di *Graph-Relational Object Queries*, linguaggio dichiarativo progettato per interrogare raccolte di documenti JSON, anche in ambienti con schema minimo o assente. 
    Con GROQ è possibile filtrare, ordinare, unire (join) documenti differenti e strutturare la risposta esattamente secondo le esigenze dell’applicazione.  
    
\end{description}

\section*{H}
\begin{description}
    %\item[Hackathon] \textbf{Hackathon} Evento intensivo, spesso di breve durata (24-48 ore), in cui team di sviluppatori e designer collaborano per creare prototipi o soluzioni innovative.
    \item[Headless] Architettura in cui il frontend è separato dal backend (ad esempio in un CMS), permettendo maggiore flessibilità e integrazione multicanale.
    \item[Handover] Termine inglese che indica il trasferimento strutturato di responsabilità, informazioni e attività da una persona o da un team a un altro.

\end{description}

\section*{I}
\begin{description}
    %\item[IaC] \textbf{Infrastructure as Code} Pratica di gestire e configurare infrastrutture tramite file di codice, versionabili e automatizzabili, invece che con operazioni manuali.
    \item[Issue] Problema, difetto o ostacolo identificato durante lo svolgimento di un progetto o di un’attività, che necessita di analisi e risoluzione. Le issue vengono generalmente tracciate tramite strumenti di gestione per monitorarne lo stato, la priorità e la responsabilità, contribuendo al miglioramento continuo e al controllo della qualità.

\end{description}

\section*{J}
\begin{description}
    \item[JSON] \textbf{JavaScript Object Notation} formato leggero e leggibile per rappresentare dati strutturati tramite coppie *chiave–valore* e array.  
    È ampiamente utilizzato per lo scambio di dati tra applicazioni, in particolare nelle API web, grazie alla sua semplicità e compatibilità con molti linguaggi di programmazione.

\end{description}

\section*{K}
%\begin{description}
    %\item[IaC] \textbf{Infrastructure as Code} Pratica di gestire e configurare infrastrutture tramite file di codice, versionabili e automatizzabili, invece che con operazioni manuali.
    %\item[Issue] Problema, difetto o ostacolo identificato durante lo svolgimento di un progetto o di un’attività, che necessita di analisi e risoluzione. Le issue vengono generalmente tracciate tramite strumenti di gestione per monitorarne lo stato, la priorità e la responsabilità, contribuendo al miglioramento continuo e al controllo della qualità.

%\end{description}

\section*{L}
\begin{description}
    %\item[IaC] \textbf{Infrastructure as Code} Pratica di gestire e configurare infrastrutture tramite file di codice, versionabili e automatizzabili, invece che con operazioni manuali.
    %\item[Issue] Problema, difetto o ostacolo identificato durante lo svolgimento di un progetto o di un’attività, che necessita di analisi e risoluzione. Le issue vengono generalmente tracciate tramite strumenti di gestione per monitorarne lo stato, la priorità e la responsabilità, contribuendo al miglioramento continuo e al controllo della qualità.
    \item[LLM] Acronimo di *Large Language Model*, modello di intelligenza artificiale basato su reti neurali di grandi dimensioni, addestrato su vasti insiemi di testi per comprendere e generare linguaggio naturale.  
    Gli LLM sono in grado di eseguire compiti come traduzione, riassunto, generazione di codice, risposta a domande e supporto alla scrittura. 
\end{description}

\section*{M}
\begin{description}
    \item[Meeting] Incontro (fisico o virtuale) tra partecipanti con uno scopo definito (allineamento, decisione, pianificazione); solitamente ha un'agenda, una durata prevista e produce output (verbale, action items).
    \item[Milestone] Evento o traguardo significativo all’interno di un progetto, utilizzato per segnare il completamento di una fase importante o il raggiungimento di un obiettivo intermedio. Le milestone aiutano a monitorare l’avanzamento, pianificare le attività successive e valutare il rispetto delle tempistiche complessive.
    %\item[Mentorship] \textbf{Mentorship} Relazione in cui una persona più esperta guida e supporta la crescita professionale di un collega meno esperto.
    %\item[Microservizi] \textbf{Microservizi} Architettura software in cui le funzionalità di un’applicazione sono suddivise in servizi indipendenti che comunicano tra loro tramite API.
    %\item[Monitoring] \textbf{Monitoring} Attività di osservazione costante di sistemi e applicazioni per garantire disponibilità, prestazioni e sicurezza.
    %\item[Manutenzione reattiva] \textbf{Manutenzione reattiva} Approccio alla manutenzione che interviene solo dopo il verificarsi di un guasto o problema.
    %\item[MTTR] \textbf{Mean Time To Recovery/Repair} Indicatore che misura il tempo medio necessario per ripristinare un servizio dopo un guasto.
\end{description}

\section*{O}
\begin{description}
    \item[Overwhelming] Termine che descrive una situazione, un’emozione o una quantità di informazioni così intensa o complessa da risultare difficile da gestire o da elaborare.  
    Può riferirsi a un carico di lavoro eccessivo, a emozioni molto forti o a un contesto che supera le capacità di risposta immediate di una persona o di un sistema.
\end{description}

\section*{P}
\begin{description}
    %\item[Project manager] \textbf{Project Manager} Figura responsabile della pianificazione, esecuzione e chiusura di un progetto: coordina risorse, gestisce tempi, costi e rischi, mantiene comunicazione con gli stakeholder e assicura il raggiungimento degli obiettivi.
    %\item[Pair programming] \textbf{Pair Programming} Tecnica di sviluppo in cui due programmatori lavorano insieme alla stessa postazione: uno scrive il codice (driver), l’altro lo rivede in tempo reale (observer).
    %\item[Patch] \textbf{Patch} Correzione software rilasciata per risolvere bug, vulnerabilità o migliorare funzionalità.
    \item[Pipeline] Sequenza automatizzata di fasi che portano il codice sorgente dalla scrittura allo sviluppo, test, integrazione e distribuzione.
    \item[PoC] \textbf{Proof of Concept} Prototipo o esperimento preliminare volto a dimostrare la fattibilità tecnica o commerciale di un’idea o soluzione.
    %\item[PMI] \textbf{Piccole e Medie Imprese} Categoria di imprese definita in base a numero di dipendenti e fatturato annuo; spesso con esigenze e budget diversi dalle grandi aziende.
    \item[Production] Ambiente operativo in cui un’applicazione è effettivamente disponibile e utilizzata dagli utenti finali.
    \item[Playground] Ambiente di test o sperimentazione interattivo che consente di eseguire, modificare e verificare in tempo reale codice, query o configurazioni, senza influire sull’ambiente di produzione. È utilizzato per apprendere funzionalità, validare prototipi e condurre esperimenti in modo rapido e controllato.
\end{description}

\section*{R}
\begin{description}
    %\item[Rollout] \textbf{Rollout} Processo di distribuzione e attivazione di una nuova funzionalità o servizio; può essere eseguito in modo graduale (phased rollout, canary release, feature flags) per minimizzare rischi e permettere rollback controllati.
    \item[RAG] \textbf{Retrieval-Augmented Generation} Approccio nell'intelligenza artificiale che combina un modulo di recupero (retriever) di documenti o frammenti rilevanti da una base di conoscenza con un modello generativo che produce la risposta finale basandosi sulle informazioni recuperate. Vantaggi: risposte più informate e aggiornabili; svantaggi: qualità dipendente dal retrieval e dalla gestione delle fonti.
    \item[Repository] \textbf{Repo} Archivio centralizzato dove viene conservato e versionato il codice sorgente di un progetto.
    %\item[Release] \textbf{Release} Rilascio pianificato di una versione del software, solitamente accompagnato da note di rilascio che descrivono nuove funzionalità e fix.
\end{description}

\section*{S}
\begin{description}
    %\item[Storefront] Interfaccia pubblica di un negozio online: pagine di prodotto, categorie, carrello e checkout — ovvero il "volto" e l'esperienza cliente dell'e-commerce.
    \item[Slack] Piattaforma di messaggistica aziendale in tempo reale per la comunicazione dei team, con canali, messaggi diretti, condivisione file e integrazioni con altri strumenti.
    \item[Store] Punto di vendita o repository: può indicare un negozio fisico, uno store online (la parte che espone e vende prodotti agli utenti) oppure, in contesti software, un archivio/repository dove sono conservati pacchetti o risorse.
    %\item[Serverless] \textbf{Serverless} Modello di esecuzione in cloud in cui l’infrastruttura è gestita dal provider e gli sviluppatori si concentrano solo sul codice, pagando a consumo.
    \item[Script] Sequenza di comandi automatizzati che eseguono operazioni ripetitive (ad esempio build, deploy, manutenzione).
    %\item[Staging] \textbf{Staging} Ambiente di prova che replica la produzione e serve a testare le modifiche prima del rilascio.
    \item[Stand-up meetings] Breve riunione quotidiana (tipica di Scrum) in cui i membri del team condividono progressi, piani e ostacoli.
    %\item[Sviluppo locale] \textbf{Sviluppo locale} Attività di sviluppo e test svolta sulla macchina del programmatore, prima di caricare il codice su repository condivisi o ambienti remoti.
    %\item[Security-by-Design] \textbf{Security-by-Design} Principio secondo cui la sicurezza deve essere integrata fin dalle prime fasi di progettazione di un sistema, non aggiunta a posteriori.
\end{description}

\section*{T}
\begin{description}
    \item[Ticket] Registro/formalizzazione di una richiesta, problema o attività (supporto, sviluppo, manutenzione); contiene descrizione, priorità, assegnatario, stato e storico delle azioni intraprese.
    \item[Testing] Insieme di attività volte a verificare che un sistema o componente soddisfi i requisiti e funzioni correttamente. Tipologie comuni: unit testing, integration testing, system testing, end-to-end (E2E), acceptance testing e regression testing. Può essere manuale o automatizzato.
    \item[Task] Attività o unità di lavoro definita all’interno di un progetto, con un obiettivo chiaro, una durata stimata e spesso una persona o un team assegnato. I task costituiscono gli elementi operativi della pianificazione e servono a monitorare l’avanzamento verso il completamento delle milestone o dei deliverable.
    \item[Trade-off] Situazione in cui occorre trovare un equilibrio tra fattori o obiettivi contrastanti, poiché il miglioramento di uno implica spesso la riduzione di un altro.  

    %\item[Time-to-market] \textbf{Time-to-Market} Tempo che intercorre tra l’ideazione di un prodotto/servizio e la sua effettiva disponibilità sul mercato.
\end{description}

\section*{U}
\begin{description}
    %\item[UML] \textbf{Unified Modeling Language} In ingegneria del software, UML (Unified Modeling Language) è un linguaggio di modellazione e specifica basato sul paradigma object-oriented. Viene usato per descrivere soluzioni analitiche e progettuali con diagrammi formali (classi, casi d'uso, sequenze, ecc.).
    %\item[Upskilling] \textbf{Upskilling} Processo di aggiornamento mirato delle competenze, in particolare tecniche e digitali, per adattarsi a nuove esigenze lavorative o tecnologiche.
    \item[User Experience] \textbf{UX} indica l’esperienza complessiva percepita da un utente durante l’interazione con un prodotto, servizio o sistema digitale.  
    Comprende aspetti come usabilità, accessibilità, efficienza, estetica e soddisfazione dell’utente.  
\end{description}



\section*{V}
\begin{description}
    \item[Versioning] Gestione e controllo delle versioni di software o documenti, che permette di tracciare le modifiche, collaborare in modo ordinato, ripristinare stati precedenti e rilasciare aggiornamenti.
\end{description}


\section*{W}
\begin{description}
    \item[Webhook] Meccanismo di comunicazione tra applicazioni che permette l’invio automatico di dati o notifiche a un endpoint specifico (URL) quando si verifica un determinato evento.  
\end{description}


\section*{X}
% %% Nessuna voce ancora per X


\section*{Y}
% %% Nessuna voce ancora per Y


\section*{Z}
% %% Nessuna voce ancora per Z